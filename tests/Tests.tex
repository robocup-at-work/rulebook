The actual competition tasks are defined as a set of so-called tests. A competition is organized in several stages. For each stage, the Technical Committee will select a set of tests to be performed by each team.
\par
A possible stage design would be as follows:

Stage 1:
\begin{itemize}
	\item BNT: Basic Navigation Test
	\item BMT: Basic Manipulation Test
	\item BTT: Basic Transportation Test
\end{itemize}


Stage 2:
\begin{itemize}
	\item CBT: Conveyor Belt Test
	\item PPT: Precision Placement Test
\end{itemize}

Finals: 
\begin{itemize}
	\item A combination of tests from previous stages
\end{itemize}

Note that not necessarily the most complex tests have to be in the finals; it might make more sense to use them as early-bird benchmark challenges for things most teams are not yet able to solve. Tests for the finals should be tests which can be solved rather well by all finalists, but with different performance, in order to increase the attractiveness of the event. The ability of a team to solve more complex tests should be reflected by the prior score the team takes into the finals