% !TEX root = ../Rulebook.tex
\label{sec:Tests}

\section{General}

\subsection{Common Rules}
\label{ssec: Common Rules}


Unless stated other, the following rules apply to all test types:

\begin{itemize}
\item The order in which the teams have to perform will be determined by a draw from the OC.
\item The prep phase has a time limit of 3 minutes.
\item Teams must not hardcode information gained from runs of previous teams.
\item A single robot is used.
\item The robot must not leave the arena.
\item The maximum objects a robot is allowed to carry is 3.
\item The robot has to start and end at the respective arena location (START, FINISH).
\item The robot will get the task specification from the AC.
\item Reaching each active Service Area successfully is rewarded once with points defined in \ref{tab:InstancePoints}.
\item Service Areas count as succesfully reached as defined in section \ref{sec:ArenaDesign}.
\item Manipulation tasks count as successful as defined in \ref{ssec:GraspingObjects} and \ref{ssec:PlacingObjects}.
\item The score for this test will be calculated as defined in \ref{sec:ScoringAndRanking}.
\item Exact test specification is displayed in table \ref{tab:Instances}.
\end{itemize}


%\end{itemize}



%The actual competition contains of a set of so-called tests. 
%A test is specified in terms of it's purpose and focus, environment features and eventually manipulation objects involved. Further, a concrete specification of the task is given and the rules to be obeyed. 

%Each test has different variability dimensions. That is, which objects to be manipulated, how many locations to visit, from which height to grasp etc. The test instances for \YEAR are defined based on the general test description and can be seen in Section~\ref{sec:ScoringAndRanking}.




%Every test has some navigation to a service area involved in it. Successful navigation will be awarded in every test according to Table \ref{tab:InstancePoints}. A navigation to a service area is successful when the robot reached the service area as defined in section \ref{ssec:Navigating}. The rewarded points for navigation to a servie area will be only awarded once per service area.

\subsection{Grasping Objects} \label{ssec:GraspingObjects}

\textbf{General}

An Object counts as successfully grasped, if the robot grasps the correct Object of the correct Service Area and transport it out of the corresponding Service Area. In that case the Service Area has an infinite height.

A robot is allowed to grasps an incorrect Object as long as the Object don't leave the Service Area with his complete form. This enables a robot to grasp an Object to examine the Object with a camera. If an incorrect Object grasped and moved out of the Service Area it is counted as an incorrect Object Grasping.

The grasping process starts when the Manipulator enters the Service Area und ends when it leaves the Service Area.

If the Objects falls down from the platform, the robot drops the Object to the floor after leaving the Service Area or the Object falls on top of the Service Area from a height higher than $5\si{\centi\meter}$ while the grasping process it is counted as an Object Loss.

If the robot collides with an Object, Decoy or Container at the Manipulation Zone it is considered as a Minor Collision.

\textbf{Rotating Table}

It is explicitly NOT allowed to stop the table (e.g. by pushing the gripper into the table surface).

It is also explicitly NOT allowed to position the gripper in a way that blocks the objects 
unless it is during the grasping process of a target object and does not affect the table rotation or other objects.

If one of these both situations happen while a grasping process then no points are rewarded for the following successful grasping of the Object.

\subsection{Placing Objects} \label{ssec:PlacingObjects}

\textbf{General}

An Object counts as successfully placed, if the robot places an correct Object to a Manipulation Zone of the correct Service Area. The robot has to place the Object so that the Object is lying with his complete form inside the Manipulation Zone. The pose of the Object on the Manipulation Zone can be chosen freely by the robot.

There is a Placement Deduction that will be subtracted from the points of the successfull Grasping. The Placement Deduction will be applied at the following situations, if multiple situations happen while the same placing process the Placement Deduction will applied only once:

\begin{itemize}
	\item If the placed Object or the Manipulator touches another Object, Container or Decoy while the placing process at the Manipulation Zone 
	\item If the placed Object is not lying with his complete form within the Manipulation Zone at the end of the placement process
\end{itemize}


If one of these situations happens the teams will get \textcolor{red}{a penalty of minus} 50 points. \textcolor{red}{This penalty} is not depending on the type (arbitrary surface, PPT, Container ...) of the placement. If \textcolor{red}{multiple} Placement Deductions appears they will be only applied once (only \textcolor{red}{minus} 50 points).

The placement process starts when the Manipulator with the Object enters the Service Area and ends when the placed Object doesn't move anymore and the Manipulator has left the Service Area. In these cases the Service Area has an infinte height.


If an Object is placed from a height higher than $5\si{\centi\meter}$ to the top of the Manipulation Zone it is considered as an Object Loss and no points are given for a successfull Placing.

If a placed Object is either incorrect or the corresponding Service Area is incorrect, the Placement is considered as an Incorrect Placement. If at this Incorrect Placement a Placement Deductions occurs then no additional penalty points will be added to the Incorrect Placement penalty.

If the robot collides with an Object, Decoy or Container at the Manipulation Zone it is considered as a Minor Collision.

\textbf{Shelf}

The Placement at the upper part of the Shelf has to be in the Manipulation Zone of the upper part of the shelf and it is allowed that the Object is moving down to the lower area of the upper part of the shelf after the placement. 


\textbf{Precision Placement}

A successfull Placement of the Object into the cavity will rewarded with a successfull Precision Placement. A successfull Precions Placement is if a  correct Object falls through the correct cavity tile. If the Object is not falling down through the cavity but is stuck in the cavity or is lying at the end of the run at the correct cavity it will be rewarded with an successfull Cavity Placement. However The Object has to be lying with his complete form within the correct cavity.

If an Object is placed on the wrong cavity tile then no points are given and in this case there is also no incorrect Object Placing penalty as long as the Object has to be placed at this Precision Table. If an Object is placed on a Precision Table that shouldn't be placed there, then the incorrect Object Placement penalty will be apllied. 

There are recovery stragedies allowed at the Precision Table to put the Objects into the cavity.
For example it is allowed to poke with the Object in the Gripper to place the Object into the cavity. This  allows the use of a force sensor for placing the Object into the cavity.
It is also allowed to move the Object on the cavity as long as the Object stays on the same cavity. This ensures that a movement of a Object over the complete Table is not allowed.

 


\newpage
%% !TEX root = ../Rulebook.tex
\newpage
\section{Basic Navigation Test}

\paragraph{Purpose and Focus of the Test}
The purpose of the \iaterm{Basic Navigation Test}{BNT} is to check whether the robots can navigate well in their environment, i.e. in a goal-oriented, autonomous, robust, and safe way.
\par
As the navigation problem is in the focus of robotics research for a long time and many approaches for solving it and its subtasks (like exploration, mapping, self-localization, path planning, motion control, and obstacle avoidance) exist, the focus of this test is to demonstrate that these approaches function properly on the robots used by the teams and in the environment defined by the test.
The arena used for this test contains all elements that affect or support navigation: walls, service areas, places, arena objects, wall markers, and floor markers. In addition, obstacles may be placed in the environment.
\par

\paragraph{Scenario Environment}
The arena used for this test contains all elements that affect or support navigation: walls, service areas, places, arena objects, wall markers, and floor markers. In addition, obstacles may be placed in the environment.

\paragraph{Manipulation Objects}
This test does not include any objects for manipulation.
\paragraph{Task}
For the navigation test, a single robot is used. The robot will be sent a task specification, which is a string containing a series of triples, each of which specifies a place, an orientation, and pause duration. The robot has to move to the places specified in the task string, in the order as specified by the string, orient itself according to the orientation given, cover a place marker, pause its movement for the time in seconds as specified by the pause length, and finally leave the arena through the gate.

The task specification consists of:

\begin{itemize}
	\item A destination location, e.g. \texttt{S1}, \texttt{D2}, \texttt{T7} or \texttt{U4}
	\item An orientation (\texttt{N}, \texttt{S}, \texttt{W}, \texttt{E})
	\item A duration in seconds
\end{itemize}

The duration is always set to 3 seconds in order to make validation easier for the referees.
%
% \subsection{Complexity Options}
%
% \subsubsection{Obstacle complexity (pick one):}
%
% \begin{itemize}
% 	\item Easy obstacles (bonus factor = +0.2): There are up to two obstacles in the arena.
% 	\item Medium obstacles (bonus factor =  +0.4): There are up to three obstacles in the arena and the placement of the obstacles is harder.
% \end{itemize}
%
%
% \subsubsection{Barrier Tape complexity (bonus factor = +0.4):}
% There are up to two barrier tape obstacles in the arena.
%
% \subsubsection{Navigation complexity (pick one):}
%
% \begin{itemize}
% 	\item Easy navigation (bonus factor = + 0.0): The place marker has to be covered in such a way by the robot that at least a part of the black area covered
% 	\item Medium navigation (bonus factor = + 0.1): The place marker has to be covered in such a way by the robot that at least a small part of the black area is covered the orientation must be correct, i.e. the robot must not deviate more than 45 deg.
% 	\item Hard navigation (bonus factor = + 0.2): The place marker has to be fully covered by the robot the orientation must be correct, i.e. the robot must not deviate more than 45 deg.
% \end{itemize}
%
%
\paragraph{Rules}
The following rules have to be obeyed:

\begin{itemize}
\item The order in which the teams have to perform will be determined by a draw.
\item After the team's robot enters the arena, it must move to the places given in the task specification and assume the orientation specified after the place. The robot may reach a destination by choosing any path.
\item The robot must visit the places in the order given by the task specification. It is possible to skip a place of the task specification and continue with the next one. In cases where the robot skipped one or multiple places there may be multiple possible matchings between places reached and places specified. In that case for calculating scores the matching is taken which leads to the highest score for the team.
\item A destination is counted as reached when the robot covers the place marker as much as the complexity level demands. The orientation must not deviate more than 45 deg.
\item When a destination is reached, the robot must stop its movement for the number of seconds specified by the break.
\item The time is stopped when the robot has completed the task and left the arena. If the team cannot complete the task within the designated time, the run will be stopped.
\end{itemize}
%
%
%
% \subsection{Scoring}
%
% \begin{itemize}
% \item The team will receive 50 points for reaching a destination correctly (place and orientation) as given in the task specification and provided it stops for the time specified.
% \item The team receives a penalty of –50 points each time the robot touches an obstacle, a wall, an arena object or a service area (i.e. any contact with the environment).
% \item 50 points are awarded for completing the task specification completely correct, i.e. visiting all destinations from the task specification according to position and orientation (according to the chosen complexity level) and finally leaving the arena.
% \item The reached points of a test will be multiplied with the complexity factor that belongs to the chosen complexity level.
% \end{itemize}


% !TEX root = ../Rulebook.tex


\section{New Test Structure}
\label{sec:New Test Structure}

For the 2023 season and onwards, the TC decided to restructure the benchmark tests.
The main changes include:

\begin{itemize}
\item The dedicated test slots for "Precise Placement" and "Rotating Table" were removed. 
PP and RT were included in the more advanced tests.
\item As this removes two competition slots from the schedule and therefore leaves some "free time", one additional transportation task was added.
\item The now four total transportation tasks were split into two categories: "basic" and "advanced".
\item The requirements for all "basic" tasks (BMT, BTT1 \& BTT2) were relaxed in order for them to suit the naming.
\end{itemize}

We think that this had the following effects:

\begin{itemize}
\item The complexity of each test now increases more linear over the competition.
\item Teams with less experience should be able to successfully participate in the competition for longer rather than coming out of a test with no points.
\item Perfect runs are more achievable (for experienced teams) in the beginner tasks,
 bringing back more relevance to bonus points for reliable and consistent performance.
\end{itemize}



% !TEX root = ../Rulebook.tex


\section{Basic Manipulation Test}
\label{sec:Basic Manipulation Test}

The \iaterm{Basic Manipulation Test}{BMT} is the initial test for all robots in \RCAW .
The main focus is to demonstrate basic object recognition and manipulation capabilities of robots.

Therefore only two service areas will be used. Those service areas will be located near to each other, e.g. WS3 and WS4 in fig. \ref{fig:arena_example} and \ref{fig:arena_map_annotated}. One service area is used as the source location and one as the target location, meaning that all objects are initially placed on the source service area and have to be delivered to the target service area.

This test involves no arbitrary surfaces, decoy objects or obstacles.
The objects that have to be manipulated are pre-selected and will include one small grey alu (f20\_20\_g), one small nut (M20) and the bolt (M20x100). The placement and orientation of the objects will be random.

%However, a total number of 5 objects are placed on the source table.
%This means that the robot has to visit each service area atleast twice to complete the test successfully (transportation limit = 3, see \ref{ssec: Common Rules}).


%
%
%\paragraph{Purpose and Focus of the Test}
%The purpose of the \iaterm{Basic Manipulation Test}{BMT} is to demonstrate basic manipulation capabilities by the robots, like grasping, turning, or placing an object.
%\par
%The focus is on the manipulation and on demonstrating safe and robust grasping and placing of objects of different size and shape. Therefore, the number of service areas will be constraint to two, one source area and one target area, which are close to each other. 
%
%\paragraph{Scenario Environment}
%Additionally to environmental elements, different manipulatable objects will be placed on the specified service areas.
%
%\paragraph{Manipulation Objects}
%The manipulation objects used in this test are defined by the instances described in Table~\ref{tab:Instances}.
%
%\paragraph{Task}
%The task consists of a sequence of grasp and place operations, with a small base movement in between. The objective is to move a set of objects from one service area into another. To complete the task the source and the target destination have to be reached at least once.
%\par
%The task specification consists of:
%\begin{itemize}
%	\item[--] The specification of the initial place
%	\item[--] A source location, given as place (any one)
%	\item[--] A destination location, given as place (any one, but nearby the source location)
%	\item[--] A list of objects to manipulated from the source to the destination service area
%	\item[--] The specification of a final place for the robot
%\end{itemize}
%
%
%\paragraph{Rules}
%The following rules have to be obeyed:
%
%\begin{itemize}
%\item The order in which the teams have to perform will be determined by a draw.
%\item The robot will get the task specification from the referee box.
%\item A service area counts as successfully reached as defined in Section~\ref{ssec:Navigating}
%\item A manipulation object counts as successfully grasped as defined in Section~\ref{ssec:GraspingObjects}
%\item A manipulation object counts as successfully placed, if the robot has placed the object into the correct destination service area as described in Section~\ref{ssec:PlacingObjects}.
%\item  The run is over when the robot reached the final place or the designated time has expired.
%\item The score for this test will be calculated as defined in \ref{sec:ScoringAndRanking}.
%\end{itemize}
%



% !TEX root = ../Rulebook.tex
\newpage
\section{Basic Transportation Test}

\paragraph{Purpose and Focus of the Test}
The purpose of the \iaterm{Basic Transportation Test}{BTT} is to assess the ability of the robots for combined navigation and manipulation tasks \robin{as well as its task planning capabilities}. 
The robots have to deal with flexible task specifications, especially concerning information about object constellations in source and target locations, and task constraints such as limits on the number of objects allowed to be carried simultaneously, etc.

\paragraph{Scenario Environment}
The arena used for this test contains all elements as for the Basic Manipulation Test. Besides that all areas may contain objects.

\paragraph{Manipulation Objects}
The manipulation objects used in this test are defined by the instances described in Table~\ref{tab:Instances}.

\paragraph{Task}
\robin{\st{A single robot is used, which is initially positioned outside of the arena near a gate to the arena.}} The task is to get several objects from the source service areas (such as \texttt{SH02}, \texttt{WS09}, or \texttt{CB02}) and to deliver them to the destination service areas (e.g. \texttt{WS11} and \texttt{SH05}). \robin{\st{Robots may carry up to three objects simultaneously.}} 
\par
The task specification consists of two lists:
The first list contains for each service area a list of manipulation object descriptions. The descriptions are similar as those used for the Basic Manipulation Test. 
The second list contains for each destination service area a configuration of manipulation objects the robot is supposed to achieve. The configuration specification is similar as used in the Basic Manipulation Test. 

The term “line” in the task specification can be ignored.

%
%\subsection{Complexity Options}
%The same complexity scoring as in BMT applies.

\paragraph{Rules}
The following rules have to be obeyed:

\begin{itemize}
 \item \robin{A single robot is used.}
 \item \robin{The robot has to start from outside the arena and to end in the final.}
\item The order in which the teams have to perform will be determined by a draw.
\item The robot will get the task specification from the referee box.
\item \robin{\st{After the team's robot starts, it must move into the arena and attempt to complete the task.}}
\item A manipulation object counts as successfully grasped as specified in Section~\ref{ssec:GraspingObjects}.
\item A manipulation object counts as successfully placed \robin{\st{, if the robot has placed the object into the correct destination service area}} as specified in Section~\ref{ssec:PlacingObjects}.
 \item \robin{A service area counts as successfully reached as defined in Section~\ref{ssec:Navigating}}
\item It is not allowed to place manipulation objects anywhere except for the robot itself and any of the available service areas.
\item A robot may carry up to three objects at the same time.
\item \robin{\st{The time is stopped when the robot has completed the task (delivered all objects to the right locations and left the arena through the exit gates). If a team cannot complete the task within the designated time, the run will be stopped.}The run is over when the robot reached the final place or the designated time has expired.}
\item \robin{The score for this test will be calculated as defined in \ref{sec:ScoringAndRanking}.}

\end{itemize}


%
%\subsection{Scoring}
%Points are awarded as follows:
%
%\begin{itemize}
%\item 75 points are awarded for successfully grasping a manipulation object required in the task specification.. 
%\item - 75 points if a wrong object has been grasped
%\item 75 points are awarded for successfully placing a manipulation object into the destination service area.
%\item 50 points are awarded for completing the task specification completely correct. 
%\item The reached points of a test will be multiplied with a defined complexity factor depending on the previously chosen complexity level
%\end{itemize}
%


% !TEX root = ../Rulebook.tex

\section{Advanced Transportation Test}
\label{sec:Advanced Transportation Test}

The \iaterm{Advanced Transportation Test}{ATT} has its origins as a BTT, but greatly increases the difficulty of the tasks to perform by introducing more random elements. 

This includes the introduction of arbitrary surfaces, shelfs and precise placement tables, barriertape as visual obstacles and the containers as target objects. With the object count further increasing, task optimization and replanning in case of failure also becomes more relevant. In addition, the Advanced Object set is introduced into the object pool.

As with the BTTs, there are two versions of the ATT, which gradually introduce the more challenging elements of \RCAW.
The following paragraphs summarize the two different levels but DO NOT override the test specification in table \ref{fig:test_specifications_instance}.

\paragraph{ATT1}
\begin{itemize}
\item Six randomly selected objects have to be transported.
\item Three randomly selected decoy objects are placed onto one or more randomly selected (active) service areas
\item There will be five active service areas (three tables, one Precise Placement and one shelf)
\item All table heights are used (0-15 $\si{\centi\meter}$).
\item One Object must be placed on a shelf (top part).
\item One object must be placed in the Precision Placement Cavity.
\item Virtual Obstacles (Barriertapes) are placed inside the arena (one blocking, one non-blocking).
\item Two service areas will have an arbitrary surface.
\end{itemize}

\paragraph{ATT2}
\begin{itemize}
\item Seven randomly selected objects have to be transported.
\item Five randomly selected decoy objects are placed onto one or more randomly selected (active)  service areas
\item There will be six active service areas (four tables, one shelf and one Rotating Table)
\item All table heights are used (0-15 $\si{\centi\meter}$).
\item One object must be picked from a shelf (lower part).
\item One object must be picked from the Rotating Table.
\item Two objects must be placed into a container (one to each color).
\item One visual and one physical obstacle is placed inside the arena (both semi-blocking).
\item Three service areas will have an arbitrary surface.
\end{itemize}

%\newpage
\section{Precision Placement Test}

\paragraph{Purpose and Focus of the Test}
The purpose of the \iaterm{Precision Placement Test}{PPT} is to assess the robot's ability to grasp and place objects into object-specific cavities. This demands advanced perception abilities (to recognize the correct cavity for each object) and manipulation abilities (to grasp and place the object in such a manner that it fits into the cavity).

\paragraph{Scenario Environment}
The same arena as for the Basic Manipulation Test is used whereas the plane of one service arena includes object-specific cavities as shown in the Figure~\ref{fig:ppt_tiles}. For each object used in the test, there will be one specific cavity. The cavity has the dimension of the object plus a 2 mm offset for each dimension. At most five cavities are used in the test.


\begin{figure} [h!]
\begin{center}
\subfloat[F20\_20]{\includegraphics[width = 2cm]{./images/ppt_F20.png}} \hspace{0.1cm}
\subfloat[S40\_40]{\includegraphics[width = 2cm]{./images/ppt_S40.png}} \hspace{0.1cm} 
\subfloat[M20\_100]{\includegraphics[width = 2cm]{./images/ppt_M20_100.png}}  \hspace{0.1cm}
\subfloat[M20]{\includegraphics[width = 2cm]{./images/ppt_M20.png}}  \hspace{0.1cm}
\subfloat[M30]{\includegraphics[width = 2cm]{./images/ppt_M30.png}}  \hspace{0.1cm}
\subfloat[R20]{\includegraphics[width = 2cm]{./images/ppt_VR20.png}} \\
\subfloat[F20\_20]{\includegraphics[width = 2cm]{./images/ppt_F20_v.png}}  \hspace{0.1cm}
\subfloat[S40\_40]{\includegraphics[width = 2cm]{./images/ppt_S40_v.png}}   \hspace{0.1cm}
\subfloat[M20\_100]{\includegraphics[width = 2cm]{./images/ppt_M20_100_v.png}} \hspace{0.1cm}
\subfloat[M20]{\includegraphics[width = 2cm]{./images/ppt_M20_v.png}}  \hspace{0.1cm}
\subfloat[M30]{\includegraphics[width = 2cm]{./images/ppt_M30_v.png}}  \hspace{0.1cm}
\subfloat[R20]{\includegraphics[width = 2cm]{./images/ppt_VR20_v.png}} 
\end{center}
\caption{Illustration of horizontal (top row) and vertical (bottom row) cavities for the different kind of manipulation objects.}
\label{fig:ppt_tiles}
\end{figure}


\paragraph{Manipulation Objects}
The manipulation objects used in this test are defined by the instances described in Table~\ref{tab:Instances}.

\paragraph{Task}
A single robot is used. The robot is placed by the team freely within the arena. The objective of the task is to pick the objects which are placed on one service area and make a precise-placement in the corresponding cavity at the service area with the special PPT platform (an example configuration is illustrated in Figure \ref{fig:ppt_plattform}). 

\begin{figure}
\centering
\includegraphics[width=0.6\textwidth ]{./images/ppt_plattform.jpg}
\caption{The PPT platform including five cavity tiles}
\label{fig:ppt_plattform}
\end{figure}

The task consists of multiple grasp and place operations, possibly with base movement in between, which will, however, be short. The task is finished once the objects are picked up and placed in the corresponding cavities or when the time foreseen for the run ends. Note that the placement of the object in the cavity is finished when the object is fallen into the cavity (i.e. at least some part of the object has to touch ground floor underneath the cavity).
\par
All objects to be transported in a run of a team and the corresponding cavities share the same orientation, either horizontal or vertical. This may vary between different teams and different runs.
\par

%
%\subsection{Complexity Levels}
%
%All Complexity Options from BMT apply.
%
%\subsubsection{PPT Orientation Complexity (bonus factor = 0.2):}
%The cavities can be placed in all orientations.
%\subsubsection{PPT Rotation Complexity (bonus factor = 0.2):}
%The cavities can be placed in all orientations.

\paragraph{Rules}
The following rules have to be obeyed:

\begin{itemize}

\item The order in which the teams have to perform will be determined by a draw.
\item The robot will get the task specification from the referee box.
\item A manipulation object counts as successfully grasped as specified in Section~\ref{ssec:GraspingObjects}.
\item An object counts as placed correctly if it fell through the correct cavity and touches the ground beyond. It may happen that an object blocks the cavity for the next object, e.g. by standing upright on the floor. In that case a referee may remove that object (which remains to count as a successful place). If the referee is not able to do so and the robot places another object into the blocked cavity, it counts as a correct placement if it would have been successful without the blocking object.
\item The time is stopped when the robot has placed the last object correctly. If a team cannot complete the task within the specified time, the run will be stopped after it is exceeded.  

\end{itemize}

%\subsection{Scoring}
%Points are awarded as follows:
%
%\begin{itemize}
%\item 50 points are awarded for successfully grasping an object.
%\item 100 points are awarded for successfully placing a manipulation object into the correct cavity.
%\item 50 points are awarded if the task specification has been completely fulfilled. The task is considered as fulfilled if all objects have been dropped in the right cavity. The robot does not have to leave the arena.
%\item a penalty of -50 points is given for each object which has been dropped into the wrong cavity
%\item The reached points of a test will be multiplied with a defined complexity factor depending on the previously chosen complexity level.
%\end{itemize}




%% !TEX root = ../Rulebook.tex
%\newpage
\section{Rotating Table Test}
\label{sec:Rotating Table Test}

The \iaterm{Rotating Table Test}{RTT} introduces moving objects to the competition.
A total of six objects are placed on the rotating table (see Section~\ref{sec:Rotating Table}), of which three must be picked and three are decoy objects.

This requires robots to detect objects and estimate their trajectory in order to grasp them successfully.
To lower the difficulty of this task, the table continously spins with a constant speed, enabling robots to use data from multiple rotations to calculate the optimal grasping position and move their manipulator in time.

%It is explicitly NOT allowed to stop the table (e.g. by pushing the gripper into the table surface).
%
%It is also explicitly NOT allowed to position the gripper in a way that blocks the objects 
%unless it is during the grasping process of a target object and does not affect the table rotation or other objects.

The table starts spinning once the runtime of each team has started.
The initial table rotation is the same for each team to ensure comparability.  

As navigation is not the focus of this test, robots only have to travel to the rotating table and do not have to move to the FINISH location. The run ends once all three objects have been picked or a rule violation has been called by the referees.

See the section \ref{ssec:GraspingObjects} for more detailed information about the grasping process at a Rotating Table.

%
%\paragraph{Purpose and Focus of the Test}
%The purpose of the \iaterm{Rotating Table Test}{RTT} is to assess the robot's ability to manipulate moving objects which are placed on a rotating turntable. The test demands fast perception and manipulation skills in order to pick up objects from a moving surface.
%
%\paragraph{Scenario Environment}
%The same arena as for the Basic Manipulation Test is used. In case that the arena does not already include such a device (see Figure \ref{fig:conveyor_belt}), it will be added only for this particular test.
%
%\begin{figure} [h!]
%	\begin{center}
%%		\subfloat[Conveyor belt]{\includegraphics[height = 4cm]{./images/conveyor_belt.jpg}} 
%%		\hspace{1cm}
%		\includegraphics[height = 6cm]{./images/rotating_table.jpg}
%	\end{center}
%	\caption{Illustration of a rotating table used in the competition.}
%	\label{fig:conveyor_belt}
%\end{figure}
%
%
%
%\paragraph{Manipulation Objects}
%The manipulation objects used in this test are defined by the instances described in Table~\ref{tab:Instances}.
%
%\paragraph{Task}
%The task of the robot is to navigate to the location of the rotating table and to grasp all objects from the moving table. The objects can pass multiple times in front of the robot, until the maximum time for the run is over. The robot is supposed to place the grasped objects on the robot itself.
%
%
%%\subsection{Complexity Options}
%%All Complexity Options from BMT apply.
%
%
%%\subsubsection{Speed Complexity (pick one):}
%%
%%\begin{itemize}
%%\item low speed (bonus factor = +0.0): The conveyor belt speed will be not more than 0.5 cm/s
%%\item medium speed (bonus factor = +02):	The conveyor belt speed will be not more than 0.75 cm/s
%%\item high speed (bonus factor = +0.4): The conveyor belt speed will be not more than 0.10 cm/s
%%\end{itemize}
%
%
%
%\paragraph{Rules}
%The following rules have to be obeyed:
%
%\begin{itemize}
%\item A single robot is used.
%\item The robot has to start from outside the arena and to end in the final.
%\item The order in which the teams have to perform will be determined by a draw.
%\item The objects are placed on the rotating table before the run starts by the OC or TC.
%\item The speed of the rotating table is determined by the OC or TC just before the test starts.
%\item The robot will get the task specification from the referee box.
%\item A service area counts as successfully reached as defined in Section~\ref{ssec:Navigating}
%\item A manipulation object counts as successfully grasped as specified in Section~\ref{ssec:PlacingObjects}.
%\item The objects have to be grasped actively from the moving table. The robot is not allowed to stop the items with its gripper.
%\item The run is over when the robot reached the final position or the designated time has expired.
%\item The score for this test will be calculated as defined in \ref{sec:ScoringAndRanking}.
%\end{itemize}
%
%
%
%%
%%\subsection{Scoring}
%%Points are awarded as follows:
%%
%%\begin{itemize}
%%\item 200 points are awarded for successfully grasping an object from the conveyor belt 
%%\item -150 points are given if an object dropped onto the ground, is placed not on the robot itself or dropped from the end of the conveyor..
%%\item -100 points are given if the referees had to switch on the belt manually after signalled by a team member, while a working automatic wireless mechanism to switch it on was available.
%%\item 50 points are awarded if the task has been fully achieved (when all objects placed on the belt are grasped).
%%\end{itemize}


\section{Final (Example)}

\subsection{Purpose and Focus of the Final}
The purpose of the Final is to show what each robot can do in combination. The Final test will be a combination of the other tests. To show how the final test could look like the following is added:

\subsection{Example Scenario}

Basically the scenario is a BTT with a PPT and Navigation Complexity included, the following adjustments have been made:

\begin{itemize}
\item Time limit is 10 min. 
\item 5 Objects will have to be transported in total. 
\item Two objects will be used twice. 
\item 3 Service- and 3 Destination-areas will be used. One service area is a PPT area.
\item The robot will have to start at the entrance, and to complete the run exit through the exit.  
\end{itemize}


\subsection{Complexity}
	
\begin{itemize}
\item Obstacle complexity (BNT)
\item Barrier Tape complexity (BNT)
\item Manipulation object complexity (BMT)
\item Decoy object complexity (BMT)
\item Orientation complexity (BMT)
\item Rotation complexity (BMT)
\item Position complexity (BMT)
\item Speed Complexity (CBT)
\item PPT Orientation Complexity (PPT)
\item PPT Rotation Complexity(PPT)
\end{itemize}

		
\subsection{Scoring}

\begin{itemize}
\item 50 points for (the first time) reaching a service or destination area. An area counts as reached when any part of the robot is within 1m of the area and the robot is oriented towards the area. The robot does not have to stand still. 
\item 50 points for grasping a (correct) object. Grasping is defined like in BTT. 
\item 50 points for transporting and putting to the correct destination area. 
\item 50 points for placing into the correct cavity of a PPT plate. 
\item If an object is placed into the wrong cavity or to the wrong area no penalty points apply. 
\item 50 points for completing the test, + time bonus points according to rules.
\item Other penalty points like losing objects etc. apply according to the BTT test, if not stated otherwise above.
\end{itemize}

.



% !TEX root = ../Rulebook.tex
\section{Test Specification Summary}
\renewcommand{\arraystretch}{1.1}
\newcommand{\R}[2]{
	\begin{turn}{90}
		\begin{minipage}[][1em][c]{#2}
		#1
	  \end{minipage}
	\end{turn}
}
\newcommand{\cir}[1]{\hspace{0.5em}\unitlength1ex\begin{picture}(2.8,2.8)%
\put(0.75,0.75){\circle{2.8}}\put(0.75,0.75){\makebox(0,0){#1}}\end{picture}}
\newcommand{\Y}{\tiny \CIRCLE}
\newcolumntype{P}[1]{>{\centering\arraybackslash}p{#1}}

\definecolor{headlineColor}{rgb}{.7,.7,.7}
\definecolor{sectionColor}{rgb}{.7,.1,.1}

\newcommand{\C}{\cellcolor{sectionColor}}
%\begin{landscape}
%\begin{table}[h!]
% \centering
% \begin{tabular}{|l|l|l*{11}{|P{1cm}}|}
%   \hhline{~~~--------}
%   \multicolumn{3}{l|}{ }                   &  \multicolumn{8}{c|}{Instances}                        \\
%   \hhline{~~~--------}
%   \multicolumn{3}{l|}{ }                   &\cir{1}&\cir{2}&\cir{3}&\cir{4}&\cir{5}&\cir{6}&\cir{7} \\
%   \multicolumn{3}{r|}{ }                   & BMT   & BTT1  & BTT2  &  BTT3 &  PPT  &  RTT  & Final  \\
%   \hhline{~~~--------} \hline
%   \multirow{5}{0.5cm}{\R{\centering Objects}{3.0cm}}
%   & \RCAW \&  RoCKIn            & atwork-commander   & 5     & 5     & 6     & 6     & 3      & 3     & 10    \\ \hhline{~----------}
%   & Decoy                       & TC   &       & 3     & 3     & 3     &        & 3     & 5     \\ \hhline{~----------}
%	 & Position                    &          & Ref   & Ref   & Ref   & Ref   & Team   & Ref   & Ref   \\ \hhline{~----------}
%	 & Rotation                    &          & Team  & Ref   & Ref   & Ref   & Team   & Team  & Ref   \\ \hhline{~----------}
%	 & Orientation                 &          & Team  & Team  & Team  & Ref   & Team   & Team  & Ref   \\ \hline
%   \multirow{6}{0.5cm}{\R{\centering Service area}{3.5cm}}
%   & Estimated Active            & atwork-commander   & 2     & 3     & 4     & 5     & 2      & 1     & 8     \\ \hline
%   & Table height                & atwork-commander   &       &       & 0 cm  &       &        &       &  0 cm \\
%   &                             &          &       &       & 5 cm  &       &        &       &  5 cm \\
%   &                             &          & 10 cm & 10 cm & 10 cm & 10 cm &  10 cm & 10 cm & 10 cm \\
%   &                             &          &       &       & 15 cm &       &        &       & 15 cm \\ \hhline{~----------}
%	 & Arbitrary surface           & TC   &       & 1     & 2     & 2     &        &       & 3     \\ \hline
%	 \multirow{3}{0.5cm}{\R{\centering Arena }{1.5cm}}
%	 & Physical Obstacles          & TC  &       &       & 2     & 2     &        &       & 2     \\ \hhline{~----------}
%	 & Virtual Obstacles           & TC  &       & 2     &       & 1     &        &       & 2     \\ \hhline{~----------}
%   &                             &          &       &       &       &       &        &       &       \\ \hhline{-----------}
%   \multirow{3}{0.5cm}{\R{\centering Grasping }{1.64cm}}
%   & Shelf unit                  & atwork-commander   &       &       &       & 2     &        &       & 2     \\ \hhline{~----------}
%	 & Rotating table          & Referee  &       &       &       &       &        & 3     & 1     \\ \hhline{~----------}
%   & Rotating direction          &          &       &       &       &       &        & Team  & Ref   \\ \hline
%   \multirow{8}{0.5cm}{\R{\centering Placement}{2.5cm}}
%   & Preisicon placement table  & atwork-commander   &       &       &       &       & 3      &       & 1     \\ \hhline{~----------}
%   & Shelf unit                  & atwork-commander   &       &       &       & 1     &        &       & 1     \\ \hhline{~----------}
%   & Red container               & atwork-commander   &       &       &       & 2     &        &       & 2     \\ \hhline{~----------}
%   & Blue container              & atwork-commander   &       &       &       & 2     &        &       & 2     \\ \hhline{~----------}
%   & Rotating turntable          & atwork-commander   &       &       & 1     &       &        &       &       \\ \hhline{~----------}
%   & Cavities Position           &    &       &       &       &       & Ref	   &       & Ref   \\ \hhline{~----------}
%   & Cavities Rotation	         &  &       &       &       &       & Ref    &       & Ref   \\ \hhline{~----------}
%   & Cavities Orientation	       &   &       &       &       &       & Team   &       & Team  \\ \hline \hline
%   \multicolumn{2}{|l|}{Duration}
%                                 & atwork-commander   & 5min  & 6min  & 10min & 10min & 4min   & 4min  & 13min \\
% 		\hline
% \end{tabular}
% \caption{Test specification in the instances of the \RCAW \YEAR competition.}
% \label{tab:Instances}
%\end{table}


\begin{figure}[h!]
	\centering
	\includegraphics[width= 1.0\textwidth ]{./images/tabels/env_table.png}
	\caption{Test specification in the environment of the \RCAW \YEAR competition.}
	\label{fig:test_specifications_environment}
\end{figure}

\begin{figure}[h!]
	\centering
	\includegraphics[width= 1.0\textwidth ]{./images/tabels/robocup_instance.jpg}
	\caption{Test specification in the instances of the \RCAW \YEAR competition.}
	\label{fig:test_specifications_instance}
\end{figure}
%\end{landscape}





%\newpage
%\section{Test Variability} \label{sec:TestVariability}

%The different optional parameters and configurations for each task are 
%mentioned in Section~\ref{sec:ArenaDesign} and \ref{sec:ManipulationTasks}. 
%Figure~\ref{fig:complexityTree} summarizes the possible variations and 
%emphasizes aspects that may be chosen.

%\begin{figure}[ht]
%\centering
%\tikzset{
  basic/.style  = {draw, rectangle, thin, text width=2cm, 
	                 rounded corners=2pt, align=center},
  root/.style   =  {basic},
  level 2/.style = {basic, sibling distance=45mm},
  level 3/.style = {basic, align=left},
  level 4/.style = {basic, align=left, fill = white!50, node distance = 0.7cm},
	box/.style = {draw, black, dashed}
}

\begin{tikzpicture}[
  font = \footnotesize,
  level 1/.style={sibling distance=70mm},
  edge from parent/.style={->,draw},
  >=latex]

% root of the the initial tree, level 1
\node[root] {Complexities}
% The first level, as children of the initial tree
  child {node[level 2] (c1) {Manipulation}
       child  {node[level 3]  (OBJECTS) {Objects}}
       child {node[level 3]  (GRASPING) {Grasping} }
       child {node[level 3, xshift =-2cm]  (PUTTING) {Putting down} }
	}
  child {node[level 2, xshift=-1cm] (ARENA) {Arena}};

%% OBJECTS
\begin{scope}[every node/.style={level 4,  top color=white!50,bottom color=gray!50,shading angle=45}]
\node [below of = OBJECTS, xshift=15pt, yshift=-20pt] (c11) {@work};
\node [below of = c11] (c12) {Rockin};
\node [below of = c12] (c13) {arbitary};
\node [below of = c13, yshift=-5pt] (c14) {number};
\end{scope}

\node [box, fit = (c11) (c12) (c13)] (BOX) {};
\node at (BOX.north west) [anchor = south west] {Type};

%% GRASPING
\begin{scope}[every node/.style={level 4}]
\node [below of = GRASPING, xshift=28pt, yshift=-20pt] (c21) {0cm};
\node [below of = c21] (c22) {10cm};
\node [below of = c22] (c23) { ??cm};
\node [below of = c23] (c24) { ??cm};
\node [below of = c24, top color=white!50,bottom color=gray!50,shading angle=45, yshift=-18pt] (c25) {Position};
\node [below of = c25, top color=white!50,bottom color=gray!50,shading angle=45] (c26) {Rotation};
\node [below of = c26, top color=white!50,bottom color=gray!50,shading angle=45] (c27) {Orientation};
\node [below of = c27, yshift=-18pt,] (c28) {red};
\node [below of = c28] (c29) { blue};
\end{scope}

\node [box, fit = (c21) (c22) (c23) (c24)] (BOX) {};
\node at (BOX.north west) [anchor = south west] {Height};

\node [box, fit = (c25) (c26) (c27)] (BOX) {};
\node at (BOX.north west) [anchor = south west] {Object Pose};

\node [box, fit = (c28) (c29)] (BOX) {};
\node at (BOX.north west) [anchor = south west] {Container};

%% ARENA
\begin{scope}[every node/.style={level 4}]
\node [below of = ARENA, xshift=15pt, yshift=-20pt] (c31) {Static};
\node [below of = c31] (c32) {Dynamic};
\node [below of = c32, yshift=-5pt] (c33) {Barrier tape};
\end{scope}

\node [box, fit = (c31) (c32)] (BOX) {};
\node at (BOX.north west) [anchor = south west] {Obstacles};

% lines from each level 1 node to every one of its "children"
 \foreach \value in {1,2,3,4}
   \draw[->] (OBJECTS.195) |- (c1\value.west);

 \foreach \value in {1,...,9}
   \draw[->] (GRASPING.195) |- (c2\value.west);
	
 \foreach \value in {1,...,4}
   \draw[->] (PUTTING.345) |- (c2\value.east);

 \foreach \value in {8,...,9}
   \draw[->] (PUTTING.345) |- (c2\value.east);
	
\foreach \value in {1,2,3}
   \draw[->] (ARENA.195) |- (c3\value.west);

\end{tikzpicture}

%\caption{Aspects of variability that may be integrated in a specific instance of a test.}
%\label{fig:complexityTree}
%\end{figure}
