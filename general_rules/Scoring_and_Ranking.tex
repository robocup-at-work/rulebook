% !TEX root = ../Rulebook.tex
\section{Scoring} \label{sec:ScoringAndRanking}

For each test the calculation of scores is defined individually, comprising points for achieving certain subtasks and penalty points.

Each test provides a set of so-called feature variations encoding the overall variability of the test (e.g. whether
obstacles can occur or not, number and type of manipulation objects). To enhance comparability among different test
runs, all teams will have to perform the same test instances as specified in Table~\ref{tab:Instances}.

For example in a \textbf{perfect} BMT run (perfect means all tasks fulfilled and no penalties given) a team could achieve 1200 points (without time bonus). This points are calculated using the values out of table \ref{tab:InstancePoints}.

\begin{itemize}
	\item 2 Service areas reached: $2 \times 25 = 50$ points
	\item Finish reached: $1 \times 50 = 50 $ points
	\item 5 Objects grasped: $5 \times 100 = 500 $ points
	\item 5 Objects placed: $5 \times 100 = 500  $ points
	\item perfect run bonus: $1 \times 100 = 100 $ points
\end{itemize}

%Explanation of the terms:
%\begin{itemize}
%\item Correct navigating is defined in Section \ref{ssec:Navigating}
%\item Correct grasping is defined in Section \ref{ssec:GraspingObjects}
%\item Correct placing is defined in Section \ref{ssec:PlacingObjects}
%\end{itemize}

\section{Time Bonus}
If a team finishes a test in the required time frame, the left over time will be added as time bonus to the points. 
Time bonus is calculated using the left over time in seconds multiplied by 10. 
\martin{formula ????}

\section{Simplifications}
Teams may use simplifications, which will result in a reduction of scores for the given run. The simplifications may be chosen per run, but need to be announced to the referees before the start of the run.

\begin{itemize}
%	\item Use of external sensors: \hfill -200 points
%	\item Use of other external objects (e.g. to support localization): \hfill -100 points
%	\item Use of own loading or unloading areas: \hfill -200 points
  \item Deactivation of Virtual Obstacles (Barrier Tape): -15\% of total points of the run
\end{itemize}

%Additional simplifications are specified for individual tests. These reductions do not count as penalty points. Teams
%that want to make use of the simplifications above have to announce them in advance of the competition to the TC. The
%TC might forbid the use of specific elements for simplification if these are not in the spirit of the league or may
%cause disproportionate advantages for a team.

\section{Penalties}
\label{sec:penalties}
Penalty points are given as follows, each time again the incident occurs:

\begin{itemize}
	\item Object Loss: \hfill -100 points
	\item Incorrect Object Placement \hfill -50 points
	\item Placement Deduction \hfill -50\%
	\item Minor collision (see Section \ref{sec:Collisions}): \hfill -50 points
	\item Major collision (see Section \ref{sec:Collisions}): \hfill -50 points and termination of the run
  \item Tape collision (see Section \ref{sec:Collisions}): \hfill -5\% of total points of current run up till
  20\%
\end{itemize}

\leander{add penalties for not successful place or at least a reference}
\lucas{add/rename penalties : Object Loss, incorrect Object Placement, Placement Deductions, Major Collision, Minor Collision, Tape Collision }

\section{Collisions}\label{sec:Collisions}

For reasons of safety of people and property it is strictly unwanted for the robot to collide
with any of the environmental objects. Only collisions of the gripper with the upside of
the service areas are allowed while a Grasping or Placing process. In all collision cases the Virtual Walls/Obstacles have an infinite height. The different kind of collisions that can occur are defined in the
following section. Any Collisions cause a point penalty that is explained in section \ref{sec:penalties}.  

\textbf{Major Collision:}

If the robot (platform and manipulator) collides with a static element of the environment or touches a Virtual Wall it is considered as a major collision. An exception of this rule is when the cables of the manipulator touches the environment while the Grasping or Placing process. As stated before a collision with the  gripper with the upside of the service areas are allowed as long as there is no fundemental change of the environment at the service area. In this case it would be considered as a major collision. This could be for example moving an arbitrary surface off the workspace. A major collision results in the termination of the run and the arena can be restored afterwards. Causing a second Major Collision in a run will end the run completely for this team.


\textbf{Minor Collision:}

If the Robot collides with Objects, Decoys or Container it is considered as a Minor Collision. 

%If the manipulator collides with an interaction element of the arena (RTT, upper level
%of Shelf) it is considered a minor collision. The only exception is the collision of the gripper
% with the surface of the service area.

\textbf{Tape Collision:}

The yellow/black tape is called Barrier Tape and represents a Virtual Obstacle. If any part of the robot
touches a barrier tape, it is considered a Tape Collision. Tape Collision induce a point penalty
proportional to the final points of the run. With each collision 5\% of the final points are deducted up to a maximum
of 20\%. For beginner teams the option exists to opt-out of Barrier Tape and take a static deduction of 15\% of the
final points of the run.

A Collions with the Marking Tape will not be penalized.

%\begin{description}
%  \item[Red/White Tape]: This tape represents a Virtual Wall with an infinitely height. This wall shall not be crossed. If any part of the robot (including the manipulator) is above the tape, it is considered a major collision. 
%  This tape is typically used to limit the area of the arena.
%  \item[Yellow/Black Tape]: This tape is called Barrier Tape and represents a Virtual Obstacle. If any part of the robot
%  touches a barrier tape, it is considered a Barrier Tape Collision. Barrier Tape collision induce a point penalty
%  proportional to the final points of the run. With each collision 5\% of the final points are deducted up to a maximum
%  of 20\%. For beginner teams the option exists to opt-out of Barrier Tape and take a static deduction of 15\% of the
%  final points of the run.
%  \item[Marking Tape]: This tape is used as an universal marker for several purposes and therefore a collision does not matter. 
%\end{description}

\begin{table}[h!]
	\caption{Definition of minor and major collisions}
	\label{tab:collisions}
	\centering
	\begin{tabular}{|l|p{1cm}|p{1cm}|p{1.5cm}|}
		\hline
		Situation                                                & Minor & Major &  Tape \\
		\hline
		Collision with static elements of arena                  &       & X     &              \\
		Collision with dynamic elements of arena                 &       & X     &              \\
		Robot Body Collision with workstation                    &       & X     &              \\
		Robot Manipulator Collision with top surface of manipulation zone       &       &       &              \\
		Robot Manipulator Cable Collision with Environment      &       &       &              \\
		Robot Manipulator Collision with Objects, Container, Decoys & X     &       &              \\
		Robot Manipulator Collision with Round Table stopping it &      & X      &              \\
		Robot Manipulator Collision with PPT Cavity surface      &      & X      &              \\
		Robot Manipulator Collision with Shelf                   &      & X      &              \\
		Virtual Obstacle (Yellow/Black Tape) Collisions          &       &       & X            \\
		Virtual Wall (Red/White Tape) Collisions                 &       & X     &              \\
		%Blue/White Tape Collision (1st time gate)                &       &       &              \\
		%Blue/White Tape Collision (2nd time gate)                &       & X     &              \\
		Marking Tape & & & \\
		\hline
	\end{tabular}
\end{table}


\section{Restarts}
Teams might use one so-called restart in a run. Scores that are received after a restart are multiplied by a factor of 0.75.
\lucas{add/rename points : correct Object Placement, successfull Precision Placement, successfull Cavity Placement}

\section{Ranking}
The tests will occur in the instances shown in Table~\ref{tab:Instances}. Ranking of the teams will be based on the sum of the achieved points over all the tests.

A team cannot get less than zero points for one run. If it is necessary to divide the competition into two stages (see section ~\ref{sec:qulaification for a test}), the scores of the first stage tests are summed up, and the teams with the highest sums proceed to the next stage (depending on overall number of teams). Depending on the number of teams and if there is enough time, all teams are allowed to enter the second stage.

In case of a tie of the first three places a deciding run will be scheduled. A point tie down from 4th place will result in to sharing the achieved place. 
\martin{wording ??}

%\setlength{\tabcolsep}{4.75pt}
%\renewcommand{\arraystretch}{1.1}
%\newcommand{\R}[2]{
%	\begin{turn}{90}
%		\begin{minipage}[][1em][c]{#2}
%		#1
%	  \end{minipage}
%	\end{turn}
%}
%
%\newcommand{\cir}[1]{\hspace{0.5em}\unitlength1ex\begin{picture}(2.8,2.8)%
%\put(0.75,0.75){\circle{2.8}}\put(0.75,0.75){\makebox(0,0){#1}}\end{picture}}
%\newcommand{\Y}{\tiny \CIRCLE}
%\newcolumntype{P}[1]{>{\centering\arraybackslash}p{#1}}
%
%\definecolor{headlineColor}{rgb}{.7,.7,.7}
%\definecolor{sectionColor}{rgb}{.7,.1,.1}
%
%\newcommand{\C}{\cellcolor{sectionColor}}


\begin{landscape}
\begin{table}
 \centering
 \begin{tabular}{|p{5.5cm}*{9}{|P{1cm}}|}
   \hhline{~--------}
   \multicolumn{1}{l|}{}                       & \multicolumn{8}{c|}{Instances}                               \\
   \hhline{~--------}
   \multicolumn{1}{l|}{}                       &\cir{1} &\cir{2} &\cir{3} &\cir{4} &\cir{5} &\cir{6} &\cir{7} \\
   \multicolumn{1}{r|}{}                       & BMT    & BTT1   & BTT2   &  BTT3  & PPT    &  RTT   & Final  \\
   \hhline{~--------}
   \hline
	 Correct service area reached                &   25   &   25   &  25    &   25   &  25    &   25   &   25   \\ \hline
   Correct object grasping           &    &    &     &    &     &        &    \\
     \hspace{0.5cm} standard            &  100   &  100   & 100    &  100   &   0    &        &  100   \\
	 \hspace{0.5cm} round table                  &        &        &        &        &        &  300   &  200   \\
	 \hspace{0.5cm} PPT area                     &        &        &        &        &        &        &  200   \\
	 \hspace{0.5cm} arbitrary surface (tbdiscussed... reduce points)           &        &  150   & 150    &  150   &        &        &  150   \\
	 \hspace{0.5cm} shelf upper level            &        &        &        &  150   &        &        &  150   \\
	 \hspace{0.5cm} shelf lower level            &        &        &        &  300   &        &        &  300   \\ \hline
   Correct object placing standard (tbdiscussed... increase, more than grasping to motivate the teams to implement placement with collision avoidance?)            &   75   &   75   &   75   &   75   &        &        &   75   \\
	 \hspace{0.5cm} PPT area                     &        &        &        &        &  200   &        &  200   \\
	 \hspace{0.5cm} shelf upper level            &        &        &        &  150   &        &        &  150   \\
	 \hspace{0.5cm} shelf lower level            &        &        &        &  150   &        &        &  150   \\ \hline
   Incorrect object placing tbdiscussed... (wrong service area?)                   & -100   & -100   & -100   & -100   & -100   &        & -100   \\
	Incorrect object placing tbdiscussed... (outside manipulation zone and/or touching other objects)                    & 50   & 50   & 50   & 50   & 50   &        & 50   \\
   Incorrect object grasping                   &        & -100   & -100   & -100   &        & -100   & -100   \\
   Completing whole task                       &   75   &  100   &  150   &  250   &   50   &   75   &  300   \\ \hline\hline
   Maximum attainable points\newline (time bonus not included)
	                                             & 1000   & 1100   & 1400   &  2000  &  700   &  1000  &  3200  \\ \hline
 \end{tabular}
 \caption{Scoring in the instances of the \RCAW \YEAR competition.}
  \label{tab:InstancePoints}
\end{table}
\end{landscape}
