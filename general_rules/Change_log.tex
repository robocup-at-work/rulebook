% !TEX root = ../Rulebook.tex

\chapter{Summary of Changes}

This chapter provides an overview for experienced teams that now the rules and just need an update on what is new for the specific year. All new teams are strongly advised to read the whole rule book thoroughly.

\section{General Rules}
\begin{itemize}
  \item The subsections "Scoring and Ranking" and "Restarts" have been moved to a new chapter (see \ref{sec:ScoringAndRanking}).
\end{itemize}

\section{Tests}
\begin{itemize}
  \item A new sub section named "Scoring and Rangking" has been added at \ref{sec:ScoringAndRanking}. The scores for many test objectives have changed including points for successful navigation as specified in \ref{ssec:Navigating} in tests other than the BNT. See Table \ref{tab:InstancePoints}
  \item The test instances have changed. For an overview see Table \ref{tab:Instances}.
  \item The descriptions and rules have been rewritten for clarity and are worth revisiting!
\end{itemize}

\subsection{Basic Manipulation Test} 
\begin{itemize}
  \item The BMT now has to be started from outside of the arena and has to be finished inside the final destination.
\end{itemize}

\subsection{Basic Transportation Test}
\begin{itemize}
  \item The different instances of the BTT have different focuses now.
  \item The BTT1 is a just a simple implementation of a Basic Transpotation Test. It is supposed to test the task planning capabilities for a very simple setup.
  \item The BTT2 is focused on different area heights. It includes all service areas of height 0, 5, 10 and 15 cm.
  \item The BTT3 is focused on more difficult manipulation and perception. It now contains 2 sets of containers that can be arbitrarily distributed among the 10 cm service areas and don't have to appear in pairs anymore. The test also contains shelf manipulation. For this year there will be different scoring for placing objects on the higher and lower levels of the shelf unit. This might change in following years. Note that the orientation of the manipulation objects is now be determined by the referees or the TC!
\end{itemize}

\subsection{Precision Placement Test}
\begin{itemize}
  \item The PPT now has to be started from outside of the arena and has to be finished inside the final.
  \item There has been added a section to the instances table \ref{tab:Instances} for determining the Position, Rotation and Orientation of the PPT tiles. For this year the orientation of the tiles may be determined by the teams. This might change in following years.
\end{itemize}

\subsection{Rotating Table Test}
\begin{itemize}
  \item The Conveyor Belt Test is now called Rotating Table Test.
  \item The conveyor belt has been completely removed from this test.
  \item The speed of the table will be determined by the OC or TC right before the test.
  \item The robot has to start from outside the arena and has to end in the final in order to be achieved a perfect run.
  \item There will be decoy objects and the position of all objects will be determined by the referees or the TC.
\end{itemize}
