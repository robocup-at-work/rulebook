% !TEX root = ../Rulebook.tex




\newpage
\section{Open Challenge}
During the Open Challenge teams are encouraged to demonstrate recent research results and the best of the robots’ abilities. It focuses on the demonstration of new approaches and applications to industrial tasks.

\subsection{Task}

The Open Challenge consists of a demonstration and an interview part. It is an open demonstration which means that the teams may demonstrate anything they prefer. The performance of the teams is evaluated by a jury consisting of all team leaders, TCs and OCs.
\begin{itemize}

\item[1.] Setup and demonstration: The team has a maximum of 10 minutes for setup, presentation and demonstration.
\item[2.] Interview: After the demonstration, there is another 5 minutes where the team answers questions by the jury members.
\end{itemize}

\subsection{Presentation}
During the demonstration, the team can present the addressed problem and the demonstrated approach.
A video projector or screen, if available, may be used to present a brief (max. 1 minute) introduction to what will be shown. The team can also visualize robot’s internals, e.g., percepts. It is important to note that the jury may decide to end the demonstration if there is nothing happening or nothing new is happening.

\subsection{Jury​ ​evaluation}
Jury​ ​of​ ​team​ ​leaders:​ All teams have to provide one person (preferably the team-leader) to follow and evaluate the entire Open Challenge. \par
Evaluation:​ both the demonstration of the robot(s), and the answers of the team in the interview part are evaluated. \par
For each of the following evaluation criteria, a maximum of tbd points is given per jury member:
\begin{itemize}
\item Overall demonstration
\item Robot autonomy in the demonstration

\item Realism and usefulness for industrial like applications (Can this be ported on real industrial scenarios?)
\item Novelty and (scientific) contribution
\item Difficulty and success of the demonstration
\end{itemize}
A jury member is not allowed to evaluate and give points for the own team. \par 
Normalization​ ​and​ ​outliers:
\begin{itemize}
\item The points given by each jury member are scaled to obtain a maximum of tbd
\item The total score for each team is the mean of the jury member scores. To neglect outliers, the N best and worst scores are left out:
\end{itemize}

\begin{equation}
score=\frac{\Sigma TeamLeaderScore}{NumberOfTeams−(2N+1)}
\end{equation} 
$N = 2$ if $NumberOfTeams \geq 10$, $1$ if $NumberOfTeams < 10$


\subsection{Team-team​ ​interaction}
Do we want to reward team-team interactions? If yes, need to be regulated somehow.

\subsection{Inter-league​ ​collaboration}
The same as above.
