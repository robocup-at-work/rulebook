\newpage
\section{Cluttering Pick}

\subsection{Purpose and Focus of the Test}
The purpose of the \iaterm{Cluttering Pick Test}{CPT} is to evaluate an interactive
mode of operation. One object is shown to the robot, it have to identify
its specific type, "search" for other instances in the environment and deliver them. The scenario is motivated by
the idea, that a human worker requires e.g. three nuts instead of one, and instructs the robot to provide them.

\subsection{Scenario Environment}
The arena used for this test contains basically all elements as for the Basic
Manipulation Test. Two service areas with a height of 10cm are involved in the test, a basic one with a white surface providing the desired object
and a second area covered by an arbitrary surface and color. Here the referees distribute
3 instances of the object in combination with decoys (see Fig.~\ref{fig:ast_example}). All geometrical definitions given in Fig.~\ref{fig:manipulation_zone} are considered here too.

\subsection{Task}
The robot starts at the defined start position outside the arena.
The task consists of navigating to the specified service area containing one instance of the
requested object. The robot localizes and identifies the object, moves to the
second service area and starts the picking operation. There will also be three to five decoy objects that must not be picked up.
\par
The task consists of a sequence of grasp operations. The objective is to pick up all 3 instance of the requested type and avoid picking other objects.
\par
The task specification consists of:
\begin{itemize}
	\item[--] The specification of the initial place
  \item[--] A location of service area containing an instance of the requested object type e.g. \texttt{WS08})
	\item[--] A source location, given as place (e.g. \texttt{WS09})
	\item[--] The specification of a final place for the robot
\end{itemize}

\paragraph{Manipulation Objects}
The manipulation objects used in this test are defined by the instances described in Table~\ref{tab:Instances}.

\subsection{Rules}
The following rules have to be obeyed:

\begin{itemize}
\item A single robot is used.
\item The robot has to start from outside the arena and to end in the final.
\item The order in which the teams have to perform will be determined by a draw.
\item At the beginning of a team's period, the team will get the task specification.
\item A service area counts as successfully reached as defined in Section~\ref{ssec:Navigating}
\item Three objects have to be picked.
\item There will be 3-5 decoy objects that must not be picked on the service area.
\item A manipulation object counts as successfully grasped as specified in Section~\ref{ssec:GraspingObjects}.
\item The run is over when the robot reached the final place or the designated time has expired.
\end{itemize}

\subsection{Scoring}
\begin{itemize}
\item 100 points are awarded for each correctly and successfully picked object
\item -50 points for every incorrectly picked object
\item 25 points for reaching the correct service area
\item 25 points for reaching the final position
\end{itemize}
